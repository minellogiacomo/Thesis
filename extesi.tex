%%%%%%%%%%%%%%%%%%%%%%%%%%%%%%%%%%%%%%%%%12pt: grandezza carattere
                                        %a4paper: formato a4
                                        %openright: apre i capitoli a destra
                                        %twoside: serve per fare un
                                        %   documento fronteretro
                                        %report: stile tesi (oppure book)
\documentclass[12pt,a4paper,openright, oneside]{report}
%
%%%%%%%%%%%%%%%%%%%%%%%%%%%%%%%%%%%%%%%%%libreria per scrivere in italiano
\usepackage[italian]{babel}
%
%%%%%%%%%%%%%%%%%%%%%%%%%%%%%%%%%%%%%%%%%libreria per accettare i caratteri
                                        %   digitati da tastiera come è à
                                        %   si può usare anche
                                        %   \usepackage[T1]{fontenc}
                                        %   però con questa libreria
                                        %   il tempo di compilazione
                                        %   aumenta

\usepackage[T1]{fontenc}
\usepackage[utf8]{inputenc}
%
%%%%%%%%%%%%%%%%%%%%%%%%%%%%%%%%%%%%%%%%%libreria per impostare il documento
\usepackage{fancyhdr}
%
%%%%%%%%%%%%%%%%%%%%%%%%%%%%%%%%%%%%%%%%%libreria per avere l'indentazione
%%%%%%%%%%%%%%%%%%%%%%%%%%%%%%%%%%%%%%%%%   all'inizio dei capitoli, ...
\usepackage{indentfirst}
%
%%%%%%%%%libreria per mostrare le etichette
%\usepackage{showkeys}
%
%%%%%%%%%%%%%%%%%%%%%%%%%%%%%%%%%%%%%%%%%libreria per inserire grafici
\usepackage{graphicx}
%
%%%%%%%%%%%%%%%%%%%%%%%%%%%%%%%%%%%%%%%%%libreria per utilizzare font
                                        %   particolari ad esempio
                                        %   \textsc{}
\usepackage{newlfont}
%
%%%%%%%%%%%%%%%%%%%%%%%%%%%%%%%%%%%%%%%%%librerie matematiche
\usepackage{amssymb}
\usepackage{amsmath}
\usepackage{latexsym}
\usepackage{amsthm}
%
\oddsidemargin=30pt \evensidemargin=20pt%impostano i margini
\hyphenation{sil-la-ba-zio-ne pa-ren-te-si}%serve per la sillabazione: tra parentesi 
					   %vanno inserite come nell'esempio le parole 
%					   %che latex non riesce a tagliare nel modo giusto andando a capo.

%
%%%%%%%%%%%%%%%%%%%%%%%%%%%%%%%%%%%%%%%%%comandi per l'impostazione
                                        %   della pagina, vedi il manuale
                                        %   della libreria fancyhdr
                                        %   per ulteriori delucidazioni
\pagestyle{fancy}\addtolength{\headwidth}{20pt}
\renewcommand{\chaptermark}[1]{\markboth{\thechapter.\ #1}{}}
\renewcommand{\sectionmark}[1]{\markright{\thesection \ #1}{}}
\rhead[\fancyplain{}{\bfseries\leftmark}]{\fancyplain{}{\bfseries\thepage}}
\cfoot{}
%%%%%%%%%%%%%%%%%%%%%%%%%%%%%%%%%%%%%%%%%
\linespread{1.3}                        %comando per impostare l'interlinea
%%%%%%%%%%%%%%%%%%%%%%%%%%%%%%%%%%%%%%%%%definisce nuovi comandi
%
\begin{document}
\begin{titlepage}                       %crea un ambiente libero da vincoli
                
%%%%%%%%%%%%%%%%%%%%%%%%%%%%%%%%%%%%%%%%
\clearpage{\pagestyle{empty}\cleardoublepage}%non numera l'ultima pagina sinistra
\end{titlepage}
\pagenumbering{roman}                   %serve per mettere i numeri romani
\chapter*{Introduzione}                 %crea l'introduzione (un capitolo
                                        %   non numerato)
%%%%%%%%%%%%%%%%%%%%%%%%%%%%%%%%%%%%%%%%%imposta l'intestazione di pagina
\rhead[\fancyplain{}{\bfseries
INTRODUZIONE}]{\fancyplain{}{\bfseries\thepage}}
\lhead[\fancyplain{}{\bfseries\thepage}]{\fancyplain{}{\bfseries
INTRODUZIONE}}
%%%%%%%%%%%%%%%%%%%%%%%%%%%%%%%%%%%%%%%%%aggiunge la voce Introduzione
                                        %   nell'indice
\addcontentsline{toc}{chapter}{Introduzione}
Le security token offerings sono un fenomeno recente che si è diffuso a partire dalla seconda metà del 2017 mantenendo inizialmente la connotazione di initial coin offerings debolmente regolamentate e solo più tardi differenziate nella loro connotazione attuale di token che si configuarano dal punto di vista giuridico come securities, al pari di altri strumenti finanziari.  

Lo scopo di questo lavoro di tesi è stato approfondire la comprensione di questo fenomeno in particolare analizzandone le caratteristiche, le cause e metodologie con le quali queste STO vengono realizzate.

In primo luogo, si definiranno le caratteristiche delle initial coin offerings e successivamente verranno individuati i criteri per identificare correttamente la differenza tra queste e le security token offerings. Una volta posti questi criteri si procederà ad analizzare i dati delle security token offerings individuate al fine di poter descrivere le metriche di questo fenomeno. Una volta valutata la rilevanza delle security token offering e le prospettive di crescita e diffusione verranno analizzate e comparate le principali implementazioni di questo modello.  



%%%%%%%%%%%%%%%%%%%%%%%%%%%%%%%%%%%%%%%%%non numera l'ultima pagina sinistra
\clearpage{\pagestyle{empty}\cleardoublepage}


\tableofcontents                        %crea l'indice
%%%%%%%%%%%%%%%%%%%%%%%%%%%%%%%%%%%%%%%%%imposta l'intestazione di pagina
\rhead[\fancyplain{}{\bfseries\leftmark}]{\fancyplain{}{\bfseries\thepage}}
\lhead[\fancyplain{}{\bfseries\thepage}]{\fancyplain{}{\bfseries
INDICE}}
%%%%%%%%%%%%%%%%%%%%%%%%%%%%%%%%%%%%%%%%%non numera l'ultima pagina sinistra
\clearpage{\pagestyle{empty}\cleardoublepage}
\listoffigures                          %crea l'elenco delle figure
%%%%%%%%%%%%%%%%%%%%%%%%%%%%%%%%%%%%%%%%%non numera l'ultima pagina sinistra
\clearpage{\pagestyle{empty}\cleardoublepage}
\listoftables                           %crea l'elenco delle tabelle
%%%%%%%%%%%%%%%%%%%%%%%%%%%%%%%%%%%%%%%%%non numera l'ultima pagina sinistra


\clearpage{\pagestyle{empty}\cleardoublepage}

\chapter{Differenze tra Initial Coin Offering e Security Token Offering}                %crea il capitolo
%%%%%%%%%%%%%%%%%%%%%%%%%%%%%%%%%%%%%%%%%imposta l'intestazione di pagina
\lhead[\fancyplain{}{\bfseries\thepage}]{\fancyplain{}{\bfseries\rightmark}}
\pagenumbering{arabic}
%mette i numeri arabi
Questo \`e il primo capitolo.

\section{Il passaggio da moneta elettronica a cryptocurrency}                 %crea la sezione

\section{Classificazione dei token nelle token sales}

\subsection{Il modello delle initial coin offering}
\subsection{Il modello delle security token offering}

\chapter{Analisi delle Security Token Offering}     
\section{Analisi dei dati}
\section{Tendenze e previsioni}
\clearpage{\pagestyle{empty}\cleardoublepage}
\chapter{Confronto tra le implementazioni}
\section{Implementazione con token ERC-20}
\section{Implementazione con }
\section{Implementazione con}
\section{Comparazione delle implementazoni}

\chapter{Proof of Concept}



%%%%%%%%%%%%%%%%%%%%%%%%%%%%%%%%%%%%%%%%%per fare le conclusioni
\chapter*{Conclusioni}
%%%%%%%%%%%%%%%%%%%%%%%%%%%%%%%%%%%%%%%%%imposta l'intestazione di pagina
\rhead[\fancyplain{}{\bfseries
CONCLUSIONI}]{\fancyplain{}{\bfseries\thepage}}
\lhead[\fancyplain{}{\bfseries\thepage}]{\fancyplain{}{\bfseries
CONCLUSIONI}}
%%%%%%%%%%%%%%%%%%%%%%%%%%%%%%%%%%%%%%%%%aggiunge la voce Conclusioni
                                        %   nell'indice
\addcontentsline{toc}{chapter}{Conclusioni} Queste sono le
conclusioni.\\
In queste conclusioni voglio fare un riferimento alla
bibliografia: questo \`e il mio riferimento \cite{K3,K4}.
%%%%%%%%%%%%%%%%%%%%%%%%%%%%%%%%%%%%%%%%%imposta l'intestazione di pagina
\renewcommand{\chaptermark}[1]{\markright{\thechapter \ #1}{}}
\lhead[\fancyplain{}{\bfseries\thepage}]{\fancyplain{}{\bfseries\rightmark}}
\appendix                               %imposta le appendici
\chapter{Prima Appendice}               %crea l'appendice
In questa Appendice non si \`e utilizzato il comando:\\
%%%%%%%%%%%%%%%%%%%%%%%%%%%%%%%%%%%%%%%%%\verb"" è equivalente all'
                                        %   ambiente verbatim,
                                        %   ma si utilizza all'interno
                                        %   di un discorso.
\verb"\clearpage{\pagestyle{empty}\cleardoublepage}", ed infatti
l'ultima pagina 8 ha l'intestazione con il numero di pagina in
alto.
%%%%%%%%%%%%%%%%%%%%%%%%%%%%%%%%%%%%%%%%%imposta l'intestazione di pagina
\rhead[\fancyplain{}{\bfseries \thechapter \:Prima Appendice}]
{\fancyplain{}{\bfseries\thepage}}



\begin{thebibliography}{90}             %crea l'ambiente bibliografia
\rhead[\fancyplain{}{\bfseries \leftmark}]{\fancyplain{}{\bfseries
\thepage}}
%%%%%%%%%%%%%%%%%%%%%%%%%%%%%%%%%%%%%%%%%aggiunge la voce Bibliografia
                                        %   nell'indice
\addcontentsline{toc}{chapter}{Bibliografia}
%%%%%%%%%%%%%%%%%%%%%%%%%%%%%%%%%%%%%%%%%provare anche questo comando:
%%%%%%%%%%%\addcontentsline{toc}{chapter}{\numberline{}{Bibliografia}}
\bibitem{K1} Primo oggetto bibliografia.
\bibitem{K2} Secondo oggetto bibliografia.
\bibitem{K3} Terzo oggetto bibliografia.
\bibitem{K4} Quarto oggetto bibliografia.
\end{thebibliography}

\end{document}
