%%%%%%%%%%%%%%%%%%%%%%%%%%%%%%%%%%%%%%%%%per fare le conclusioni
\chapter*{Conclusioni}
%%%%%%%%%%%%%%%%%%%%%%%%%%%%%%%%%%%%%%%%%imposta l'intestazione di pagina
\rhead[\fancyplain{}{\bfseries
CONCLUSIONI}]{\fancyplain{}{\bfseries\thepage}}
\lhead[\fancyplain{}{\bfseries\thepage}]{\fancyplain{}{\bfseries
CONCLUSIONI}}
%%%%%%%%%%%%%%%%%%%%%%%%%%%%%%%%%%%%%%%%%aggiunge la voce Conclusioni
                                        %   nell'indice
\addcontentsline{toc}{chapter}{Conclusioni} 

Le STOs  rappresentano  una soluzione tecnologica interessante che ha ampie prospettive di espansione. In conclusione, l'analisi di questo fenomeno ha chiarito alcuni aspetti sul perché le STOs siano nate e perché si siano diffuse; grazie all'approfondimento delle tecniche di implementazione inoltre si è potuto delineare un panorama complessivo delle funzionalità che le STOs offrono e i differenti casi d'uso in cui risulta possibile utilizzarle. 

Nonostante la scarsità dei dati disponibili, dovuti sia alla modernità delle STOs che al valore economico degli stessi dati, è stato dimostrato come un'analisi empirica, seppur limitata, può chiarire alcune tendenze osservabili in questo mercato. Tuttavia sarebbe azzardato considerare l'analisi effettuata come soddisfacente: è auspicabile che in lavori futuri sia possibile raccogliere una quantità di dati maggiore. 

Grazie alle funzionalità di cui dispongono, le STO e le IEOs, che altro non sono che un particolare sottoinsieme delle STOs, hanno la potenzialità di espandersi in un mercato molto amplio. In particolare, la spinta legislativa dovuta a fattori esterni e la facilità di esecuzione offerta dalle IEOs fanno presumere che questa modalità di raccolta di capitale continuerà a diffondersi, anche se non è possibile sostenere l'affermazione con dati empirici. 